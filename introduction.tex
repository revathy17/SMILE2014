\section{Introduction}
\label{sec:Introduction}

%use of twitter in crisis management
The power of social media was demonstrated during Hurricane Sandy when more than 20 million tweets related to the hurricane were posted in a span of three days. It was reported that the volume of tweets doubled in these days as compared to the previous two days. 35\% of these tweets contained news from media channels, information from government sources and eyewitness accounts\footnote{\url{http://www.journalism.org/2012/11/06/hurricane-sandy-and-twitter/}}. This extensive use of social networking platforms has paved the way for new areas of research which focus on the use of these platforms in disaster management~\cite{purohit2013emergency}. On the other hand, only  3.17\% of tweets are tagged with geographic coordinates~\cite{morstatter2013sample}. In improving emergency response, using information from Twitter, the location of an online user plays an important role. 

Current approaches to detect location based on the content of their tweets, focus on building statistical models using a training dataset of tweets.
%Pavan makes a point: "During real-world events, this requirement turns to a challenge, specifically for content-based approaches." But I dont agree. If you already have a model trained for a given location then this argument does not hold so deleting it.
In the event of a disaster, to identify user location in real time, we need an approach that can be easily adapted to any geographic location. To this end, we present an approach that utilizes Wikipedia as a source of knowledge to predict users' location based on their online content. Briefly, our approach uses the graph structure of Wikipedia to find entities with a local geographic scope. The presence of these entities in users' tweets help estimate their location. Our intuition is that, more the users talk about entities with a local geographic scope, more are their chances of belonging to that location. Preliminary evaluation of our approach with a random sample from the dataset shared by Cheng et al\cite{cheng2010you} has shown promise and performs better than their baseline.   
%discuss words/entities with Pavan

In the rest of this paper, we will first discuss the related work on location prediction of Twitter users in Section~\ref{sec:RelatedWork}, followed with a detailed explanation of our approach in Section~\ref{sec:approach}. Section~\ref{sec:evaluation} discusses a preliminary evaluation of our approach and the paper concludes with discussion future directions we plan to take in Section~\ref{sec:conclusion}. 