\section{Introduction}
\label{sec:Introduction}

%use of twitter in crisis management
People turned to Twitter during "Hurricane Sandy`` to communicate, express, share and receive information about the disaster. Twitter reported that its users had generated more than 20 Million tweets regarding the hurricane in a span of three days. It is also reported that the volume of tweets spiked compared to other days and most of these tweets (close to 35\%) contained either news or information regarding the disaster\footnote{\url{http://www.journalism.org/2012/11/06/hurricane-sandy-and-twitter/}}. These enourmous utilization of social networking platforms during disasters has opened up a new research focus to utilize the platforms to manage and assist people during disaster~\cite{purohit2013emergency}. 

On the other hand, only 3.17\% tweets generated are taged with geographical information~\cite{morstatter2013sample}. In order to assist and manage disasters by relying on Twitter information, location of users plays a prominent role. For instance, a user with the following tweet "Need based tweet for food?" who is in need of information related to RedCross camps near by will have higher chances of getting required information if his/her location is known.  

Existing approaches to location prediction of users are either network based{cite} or supervised content-based{}. Both approaches require prior data either about the users (user's network) or training data for each location (supervised content-based). During real-world events, this requirement turns to a challenge, specifically for content-based approaches. In order to overcome this issue, in this work we present an approach that utilizes Wikipedia as a source of knowledge to predict users' location based on the content generated. Briefly, our approach uses the graph structure of Wikipedia to find local entities for each location. The presence of these entities in users' tweets help determine/predict their locations. Our intuition is that, more the users talk about local entities of a particular location, most likely they are from the corresponding location. Preliminary evaluation of our approach with a random sample from the dataset shared by prev approach[cite] has shown promise and performs better than the baseline.   

In the rest of this paper, we will first discuss the related work on location prediction on Twitter in Section~\ref{sec:relatedwork}, followed with a detailed explanation of our approach in Section~\ref{sec:approach}. Section~\ref{sec:evaluation} discusses a preliminary evaluation of our approach and the paper concludes with discussion future directions we plan to take in Section~\ref{sec:conclusion}. 
 
%There exists supervised approaches that predict users' location based on users' tweets~\cite{workwork,work}, these approaches would fail during dynamic situations due to the unavailability of training datasets.  In this work, we present a real-time approach for detecting users' locations on Twitter by leveraging Wikipedia as the background knowledge. The approach is content-based to predict the location.    
 
%Social Networking platforms such as Twitter has been overflooded with user generated content during disasters. For instance, during "Hurricane Sandy`` Twitter reported that more than 20 million tweets related to the hurricane were posted in a span of three days\footnote{\url{http://www.journalism.org/2012/11/06/hurricane-sandy-and-twitter/}}. Individuals and organizations have both turned to Twitter to coordinate relief efforts. Mining these tweets can provide valuable information to assist people in a timely manner~\cite{purohit2013emergency}. To make these tweets informative and actionable, identifying the originating location of the tweet (and hence the user) is very important. \cite{morstatter2013sample} showed that only 3.17\% of tweets are tagged with geographical information. Thus, geo tagging tweets is an important problem to solve.

%Current approaches to detect location based on the content of tweets focus on building statistical models using a training dataset of tweets. In this paper we propose an approach that uses Wikipedia as a knowledge base to identify words with a local geographic scope. Our objective is to show that words having a strong association with a particular location, can be determined using the information available in Wikipedia.

%Wikipedia is a large encyclopedia containing dedicated pages for cities. Proportional to the size of the city, its Wikipedia page generally contains a variety of information about the city like its geography, culture, sports teams, cityscape, transportation etc.The information is very detailed for a bigger city like Paris\footnote{http://en.wikipedia.org/wiki/Paris} (population of 12 M) and significantly lower for a smaller county like Monroe County, Wisconsin\footnote{http://en.wikipedia.org/wiki/Monroe\_County,\_Wisconsin} (population of 44 K). 

%Our hypothesis is that by spotting entity in tweets and correlating them with the occurrence of Wikipedia entities for a given city, we can identify the location of a user. As in statistical methods, we do not need to manually collect seed words for each city of interest nor do we need a large training dataset to select the local words.

