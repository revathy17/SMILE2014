\section{Introduction}
\label{sec:Introduction}

%use of twitter in crisis management
The power of social media was demonstrated during Hurricane Sandy when more that 20 million tweets related to the hurricane were posted in a span of three days. Individuals and organizations have both turned to Twitter to coordinate relief efforts. Mining these tweets can provide valuable information to assist people in a timely manner\cite{purohit2013emergency}. To make these tweets informative and actionable, identifying the originating location of the tweet (and hence the user) is very important. \cite{morstatter2013sample} showed that only 3.17\% of tweets are tagged with geographical information. Thus, geo tagging tweets is an important problem to solve.

Current approaches to detect location based on the content of tweets focus on building statistical models using a training dataset of tweets. In this paper we propose an approach that uses Wikipedia as a knowledge base to identify words with a local geographic scope. 

Wikipedia is a large encyclopedia containing dedicated pages for cities. Proportional to the size of the city, its Wikipedia page generally contains a variety of information about the city like its geography, culture, sports teams, cityscape, transportation etc. Our hypothesis is that by spotting entity in tweets and correlating them with the occurrence of Wikipedia entities for a given city, we can identify the location of a user. 





