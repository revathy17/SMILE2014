\section{Introduction}
\label{sec:Introduction}

%use of twitter in crisis management
The power of social media was demonstrated during Hurricane Sandy when more than 20 million tweets related to the hurricane were posted in a span of three days. It was reported that the volume of tweets doubled in these days as compared to the previous two days. 35\% of these tweets contained news from media channels, information from government sources and eyewitness accounts\footnote{\url{http://www.journalism.org/2012/11/06/hurricane-sandy-and-twitter/}}. This kind of extensive use of social networking platforms during emergency situations has paved the way for new areas of research which focus on leveraging these platforms for disaster management~\cite{purohit2013emergency}. 

In improving emergency response using information from Twitter, the location of an online user plays an important role. However, only  3.17\% of tweets are tagged with geographic coordinates~\cite{morstatter2013sample}. As tweets are generally informal in nature and contain many acronyms and slang words, researchers have focussed on using statistical methods for identification of words with a strong geographic scope and then use these words to identify the location of a user.
%Current approaches to predict location of Twitter users based on their tweets, focus on building statistical models.
In the event of a disaster, to identify user location in real time, we need an approach that can be easily adapted to any geographic location. In order to overcome this challenge, we present an approach that utilizes Wikipedia as a source of background knowledge to predict users' location based on their online content. Briefly, our approach uses the graph structure of Wikipedia to find entities with a local geographic scope. The presence of these entities in users' tweets help estimate their location. Our intuition is that, more the users talk about entities with a local geographic scope, more are their chances of belonging to that location. Preliminary evaluation of our approach with a random sample from the dataset shared by Cheng et al\cite{cheng2010you} has shown promise and performs better than their baseline.   

In the rest of this paper, we will first discuss the related work on location prediction of Twitter users in Section~\ref{sec:RelatedWork}, followed by a detailed explanation of our approach in Section~\ref{sec:approach}. Section~\ref{sec:evaluation} discusses a preliminary evaluation of our approach and the paper concludes with a discussion on future work in Section~\ref{sec:conclusion}. 