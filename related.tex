\section{Related Work}
\label{sec:RelatedWork}
There have been two main approaches in addressing the problem of location identification of a twitter user: \begin{inparaenum}[(1)] 
\item Using the content of the tweets: based on the premise that the online content of a user is influenced by the geographical location of the user
\item Using the network information of the user: based on the assumption that the locations of the people in a user's network and their online interaction with the user can be used to determine the user's location.
\end{inparaenum}

Content-based location detection relies on a significantly large training dataset to build a statistical model that identifies words with a local scope. Cheng et al. \cite{cheng2010you} proposed a probabilistic framework for estimating a Twitter user's city-level location based on the content of approximately 1000+ tweets of each user. The locality of terms was determined by its spatial variation across the United States. Their approach on a test dataset of 5119 users, could locate 51\% of the users within 100 miles and the average error distance was reported as 535.564 miles. The disadvantage of this approach was the assumption that a "term" is spatially significant to only one location/city. This challenge was addressed by Chang et al.~\cite{chang2012phillies} by modeling the variations as a Gaussian mixture model. Their tests on the same dataset showed an accuracy (within 100 miles) of 0.499 with 509.3 miles of average error distance.
\cite{ferrara2012web} created language models at different granularity levels from zip code to country level using a training dataset of 5.8 million geotagged tweets. At the city-level, they reported an accuracy of 65.7\% and 29.8\% on two different datasets.
 %They reported their results on two datasets - SPRITZER containing 5\% of the public twitter stream of 4 weeks and FIREHOSE containing 700,000 tweets from the Twitter Firehose. At the city-level, they reported an accuracy of 65.7\% and 29.8\% on the SPRITZER and the FIREHOSE dataset respectively.

Network based solutions requires the network information of a user. McGee et al. ~\cite{mcgee2013location} used the interaction between users in a network to train a Decision Tree to distinguish between pairs of users likely to live close by. They reported an average error distance of 21 miles for 80\% of their users. ~\cite{rout2013s} formulated this task as a classification task and trained an SVM classifier with features based on the information of users' followers-followees who have their location information available. They tested their approach on a random sample of 1000 users and reported 50.08\% accuracy at the city level. However, the limitation of a network-based approaches is the availability of location information of people in the given user's network.   

%Mentioned before already - this is repetition so deleting
%The above mentioned approaches require prior training dataset (of either the content or network), which can be a bottleneck during disaster management. Our goal is to overcome this requirement of training data for each city by leveraging Wikipedia as the knowledge source.
