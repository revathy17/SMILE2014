\section{Related Work}
\label{sec:RelatedWork}
There have been two main approaches in addressing the problem of location identification of a twitter user: \begin{inparaenum}[(1)] 
\item Using the content of the tweets,
\item Using the network information of the user. 
\end{inparaenum}

The first approach is based on the premise that the online content of a user is influenced by the geographical location of the user. Content-based location detection relies on a significantly large training dataset to build a statistical model that identifies words with a local scope. Use of these words in tweets are then used to narrow down the location of any user. 
%The main disadvantage of this method is that to identify tweets from a given location, it is required that we have a set of good quality of tweets from this location to train the model.
Cheng et al. \cite{cheng2010you} proposed a probabilistic framework for estimating a Twitter user's city-level location based on the content of approximately 1000+ tweets of each user. They estimate the spatial dispersion of a word and apply Laplace smoothing to overcome the sparsity of words across locations in their dataset. Their training dataset consisted of 130,689 users with 4,124,960 status updates. Their test dataset contained 5119 users with 1000+ tweets of each user. 51\%  of these users could be located down to their city-level within 100 miles. The average error distance was reported as 535.564 miles. Kinsella et al. \cite{ferrara2012web} created language models at different granularity levels from zip code to country level using a training dataset of 5.8 million geotagged tweets. They reported their results on two datasets - SPRITZER containing 5\% of the public twitter stream of 4 weeks and FIREHOSE containing 700,000 tweets from the Twitter Firehose. At the city-level, they reported an accuracy of 65.7\% and 29.8\% on the SPRITZER and the FIREHOSE dataset respectively.

A network based solution requires the network information of a given user. McGee et al. \cite{mcgee2013location} used the relationship between users on Twitter to determine their location. Their training dataset consisted of 1.6 million twitter users and their network information. On a test dataset of 249,584 users, they reported 63.9\% accuracy in determining the location within 25 miles. Rout et al. \cite{rout2013s} formulated this task as a classification task and trained an SVM classifier on twitter users with known location, to use a person's social network to locate them. Their training dataset contained 200,000 twitter users. They tested their approach on a random sample of 1000 users and reported 50.08\%accuracy at the city level.
%citeInferring the location of twitter messages based on user relationships.