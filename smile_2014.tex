%%%%%%%%%%%%%%%%%%%%%%% file typeinst.tex %%%%%%%%%%%%%%%%%%%%%%%%%
%
% This is the LaTeX source for the instructions to authors using
% the LaTeX document class 'llncs.cls' for contributions to
% the Lecture Notes in Computer Sciences series.
% http://www.springer.com/lncs       Springer Heidelberg 2006/05/04
%
% It may be used as a template for your own input - copy it
% to a new file with a new name and use it as the basis
% for your article.
%
% NB: the document class 'llncs' has its own and detailed documentation, see
% ftp://ftp.springer.de/data/pubftp/pub/tex/latex/llncs/latex2e/llncsdoc.pdf
%
%%%%%%%%%%%%%%%%%%%%%%%%%%%%%%%%%%%%%%%%%%%%%%%%%%%%%%%%%%%%%%%%%%%


\documentclass[runningheads,a4paper]{llncs}

\usepackage{amssymb}
\setcounter{tocdepth}{3}
\usepackage{graphicx}
\usepackage{graphics}
\usepackage{multirow}
\usepackage{color}
\usepackage{paralist}
\usepackage{url}
\usepackage{subfigure}
\usepackage{listings} 
\usepackage{booktabs}
\usepackage{amsmath}
\usepackage{wrapfig}

\lstloadlanguages{html,xml}
\definecolor{grey}{rgb}{0.9,0.9,0.9} 
\lstset{
        tabsize=2, 
        frame=single, 
        breaklines=true, 
        basicstyle=\footnotesize\ttfamily,
        backgroundcolor=\color{grey},
        xleftmargin=0mm,
        xrightmargin=0mm,
        captionpos=b
}

\usepackage{url}
\urldef{\mailsa}\path|{revathy,pavan,amit}@knoesis.org|    
\newcommand{\keywords}[1]{\par\addvspace\baselineskip
\noindent\keywordname\enspace\ignorespaces#1}

\graphicspath{./images/}
\DeclareGraphicsExtensions{.pdf,.jpeg,.png}	

\usepackage{color} 
\definecolor{gray}{rgb}{0.7,0.7,0.7}
\definecolor{blue}{rgb}{0.1,0.1,0.5}
\newcommand{\attention}[1]{{\color{red}\textbf{#1}}}
%\newcommand{\comment}[1]{\textit{\color{blue}#1}}
%\newcommand{\comment}[2]{{\color{blue}\textit{\textbf{#1}}(\textit{#2})}}
\newcommand{\comment}[2]{}

%\newcommand{\softdelete}[1]{{\color{gray}\textit{#1}}}
\newcommand{\softdelete}[1]{}

\newcommand{\uri}[1]{\texttt{#1}}
\newcommand{\literal}[1]{`\textit{#1}'}
\newcommand{\BibTeX}{{\sc Bib}\TeX}
\lstset{basicstyle=\small}
\begin{document}

\mainmatter  % start of an individual contribution

% first the title is needed
\title{Geo Tagging Twitter Users using Wikipedia}

% a short form should be given in case it is too long for the running head
\titlerunning{Geo Tagging Twitter Users using Wikipedia}

\author{Revathy Krishnamurthy\and Pavan Kapanipathi\and Amit Sheth}

\authorrunning{R.Krishnamurthy\and P.Kapanipathi\and A.Sheth}

\institute{
Kno.e.sis Center, CSE Department\\
Wright State University, Dayton, OH - USA \\
}

%\toctitle{Semantic Multicasting}
\tocauthor{R.Krishnamurthy, P.Kapanipathi, A.Sheth}
\maketitle

\begin{abstract}
Prior knowledge of geographical locations of Twitter users will facilitate improved crisis management. However, a recent study has shown that only 3.17\% of tweets are geotagged. Existing approaches are supervised and require training datasets to predict locations. Since crisis management is time-sensitive, the requirement of training data forms a bottleneck due to the time consuming process of creation of statistical models by crawling for tweets for each location needed. To this end, we propose an unsupervised approach that uses Wikipedia as a background knowledge to analyse tweets in order to predict the location of the users. This eliminates the need for a training dataset. We show that initial experiments beats the baselines of existing approaches with an accuracy of approximately 30\%.

%deleted:https://pressroom.usc.edu/twitter-and-privacy-nearly-one-in-five-tweets-divulge-user-location-through-geotagging-or-metadata
\end{abstract}
\section{Introduction}
\label{sec:Introduction}

%use of twitter in crisis management
The power of social media was demonstrated during Hurricane Sandy when more that 20 million tweets related to the hurricane were posted in a span of three days. Individuals and organizations have both turned to Twitter to coordinate relief efforts. Mining these tweets can provide valuable information to assist people in a timely manner\cite{purohit2013emergency}. To make these tweets informative and actionable, identifying the originating location of the tweet (and hence the user) is very important. \cite{morstatter2013sample} showed that only 3.17\% of tweets are tagged with geographical information. Thus, geo tagging tweets is an important problem to solve.

Current approaches to detect location based on the content of tweets focus on building statistical models using a training dataset of tweets. In this paper we propose an approach that uses Wikipedia as a knowledge base to identify words with a local geographic scope. 

Wikipedia is a large encyclopedia containing dedicated pages for cities. Proportional to the size of the city, its Wikipedia page generally contains a variety of information about the city like its geography, culture, sports teams, cityscape, transportation etc. Our hypothesis is that by spotting entity in tweets and correlating them with the occurrence of Wikipedia entities for a given city, we can identify the location of a user. 






\section{Related Work}
\label{sec:RelatedWork}
There have been two main approaches in addressing the problem of location identification of a twitter user: \begin{inparaenum}[(1)] 
\item Using the content of the tweets: based on the premise that the online content of a user is influenced by the geographical location of the user
\item Using the network information of the user: based on the assumption that the locations of the people in a user's network and their online interaction with the user can be used to determine the user's location.
\end{inparaenum}

Content-based location detection relies on a significantly large training dataset to build a statistical model that identifies words with a local scope. Cheng et al. \cite{cheng2010you} proposed a probabilistic framework for estimating a Twitter user's city-level location based on the content of approximately 1000+ tweets of each user. The locality of terms was determined by its spatial variation across the United States. Their approach on a test dataset of 5119 users, could locate 51\% of the users within 100 miles and the average error distance was reported as 535.564 miles. The disadvantage of this approach was the assumption that a "term" is spatially significant to only one location/city. This challenge was addressed by Chang et al.~\cite{chang2012phillies} by modeling the variations as a Gaussian mixture model. Their tests on the same dataset showed an accuracy (within 100 miles) of 0.499 with 509.3 miles of average error distance.
\cite{ferrara2012web} created language models at different granularity levels from zip code to country level using a training dataset of 5.8 million geotagged tweets. At the city-level, they reported an accuracy of 65.7\% and 29.8\% on two different datasets.
 %They reported their results on two datasets - SPRITZER containing 5\% of the public twitter stream of 4 weeks and FIREHOSE containing 700,000 tweets from the Twitter Firehose. At the city-level, they reported an accuracy of 65.7\% and 29.8\% on the SPRITZER and the FIREHOSE dataset respectively.

Network based solutions requires the network information of a user. McGee et al. ~\cite{mcgee2013location} used the interaction between users in a network to train a Decision Tree to distinguish between pairs of users likely to live close by. They reported an average error distance of 21 miles for 80\% of their users. ~\cite{rout2013s} formulated this task as a classification task and trained an SVM classifier with features based on the information of users' followers-followees who have their location information available. They tested their approach on a random sample of 1000 users and reported 50.08\% accuracy at the city level. However, the limitation of a network-based approaches is the availability of location information of people in the given user's network.   

%Mentioned before already - this is repetition so deleting
%The above mentioned approaches require prior training dataset (of either the content or network), which can be a bottleneck during disaster management. Our goal is to overcome this requirement of training data for each city by leveraging Wikipedia as the knowledge source.

 
\section{Approach}
\label{sec:approach}
Previous research \cite{cheng2010you} \cite{bo2012geolocation} have established that the content of a user's posts reflects his/her geographical location. We propose to use the information available in Wikipedia to establish words that are most representative of a given location. 

\subsection{Dataset}
We selected 1670 cities in the United States of America having population greater than 20000. We randomly selected 600 users containing 1000+ tweets each, from the dataset made publicly available by Cheng et al\cite{cheng2010you}.

\subsection{Creation of Background Knowledge}
Wikipedia is a crowd sourced encyclopedia. Links to internal Wikipedia pages from a given page are an important feature of all Wikipedia pages. The aim of these links is to increase the understanding of a user about the given page. For instance, the Wikipedia page of \textit{Boston, Massachusetts} \footnote{http://en.wikipedia.org/wiki/Boston} mentions the \textit{Boston Red Sox}, in the Sports section. It also provides a hyperlink to Boston Red Sox, that allows the user to navigate to the Wikipedia page of \textit{Boston Red Sox}. We base our approach on the assumption that these internal links share varying degrees of relevance to the Wikipedia page of the city. As in the previous example, the Wikipedia page of Boston also contains an internal link to \textit{Major League Baseball} which would be less representative of Boston than the \textit{Boston Red Sox}. 

The entire collection of Wikipedia is available for download\footnote{http://en.wikipedia.org/wiki/Wikipedia:Database\_download}. We use the dump dated 14-Feb-2014 to extract the internal links from the Wikipedia pages of all the cities in our dataset. Figure 1 shows the distribution of the count of internal links among all the city pages. From our dataset, \textit{Pittsburgh} had 2684 as the largest count of internal links and \textit{Round Lake Beach, Illinois} had 33 as the smallest count of internal links.

\subsection{Scoring City-Specific Entities}
Given a set of internal links for a city, we score each link to determine the degree of its relevance to the city. The more a given internal link is common to the cities in our dataset, the less it maybe relevant to one particular city. For example, in our dataset of 1650 cities, an internal link to the Wikipedia page of \textit{Barack Obama} appears 105 times as opposed to \textit{Southern California} and \textit{Golden Gate Bridge} which appear 50 and 6 times respectively. 

Mendes et al. \cite{mendes2011dbpedia} proposed \textit{Inverse Candidate Frequency} for the task of entity disambiguation in DBPedia Spotlight. The idea behind ICF is that "a word commonly co-occuring with many resources is less discriminative overall". We use this intuition to identify the discriminative ability of an internal link with respect to a city. Let C be the set of cities in our dataset. Let I be the set of internal links for a city c $\in$ C. The ICF of an internal link i $\in$ I, that appears in \textit{n} cities, is defined as:
\begin{equation}
	ICF(i) = \log |C|- \log n
\end{equation}

\subsection{Location Estimation}
We used Zemanta\footnote{http://www.zemanta.com} to annotate tweets. It maps entities in the input text to Wikipedia pages.

For a user U, let $T_{u}$ be the set of their tweets, $Z_{u}$ = \{$z_{1}$,$z_{2}$,...,$z_{k}$\} be the set of entities annotated by Zemanta that map to a Wikipedia url. Let |$z_{k}$| represent the cardinality $z_{k}$ in $T_{u}$. Let C be the set of cities in out dataset and $\forall$ $c_{j}$ $\in$ C, let $L$ be the set of its internal wiki links where \textit{ICF}($l_{i}$) is the score $\forall$ $l_{i}$ $\in$ L.

For the user U we compute the score of each city in our set as:
\begin{equation}
	Score(c_{j}) = \sum_{i=1}^I |l_{i}| \times ICF(l_{i})\qquad  \forall l_{i} \in Z_{u}
\end{equation}

We tag the city with the maximum score as the location of the user.





 

\section{Evaluation}
\label{sec:evaluation}

\subsection{Dataset}
From the test dataset published by \cite{cheng2010you}, we randomly selected 935 users from United States. These users were distributed across 48 states. For each user, the dataset contains approximately 1000+ tweets. To create our knowledgebase, we selected 1670 cities in the United States of America having population greater than 20000. The entire collection of Wikipedia is available for download\footnote{\url{http://en.wikipedia.org/wiki/Wikipedia:Database_download/}}. We use the dump dated 14-Feb-2014 to extract the internal links from the Wikipedia pages of all the cities in our dataset. 

\subsection{Evaluation Metrics}
We use the two metrics defined in \cite{cheng2010you} to evaluate our system \begin{inparaenum}[(1)] \item Accuracy \item Average Error Distance \end{inparaenum}. 
Accuracy is defined as the number of users identified within 100 miles of their actual location.
Error distance is the distance between the actual location of the user and the estimated location by our algorithm. Average Error Distance is the average of the error distance across all users.

\subsection{Experimental Results}
Our approach was able to locate 30.16\% of the users within 100 miles of their actual location and the Average Error Distance across the 935 users was 886.25 miles. These users were distributed across 46 states. Table \ref{table:wikilinks} shows a sample of the local words identified using Wikipedia, in the tweets of these users.

\begin{table}
\begin{tabular}{|p{3cm}|p{8cm}|}
\hline
\textbf{Location}&{\textbf{Wikipedia Links from User Tweets}}\\
\hline
{Chicago, Illinois}&{Chicago Cubs; North Center, Chicago;The Oprah Winfrey Show;Chicago White Sox}\\
\hline
{Las Vegas, Nevada}&{University of Nevada,Las Vegas; Las Vegas Boulevard; McCarran International Airport}\\
\hline
{Atlanta, Georgia}&{Atlanta Braves; Young Jeezy; Georgia Institute of Technology; Philips Arena; Buckhead (Atlanta)}\\
\hline
{Philadelphia, Pennsylvania}&{National Football League; Philadelphia Phillies; Philadelphia Eagles; Philadelphia Flyers}\\
\hline
{Detroit, Michigan}&{Eminem; General Motors; Detroit Red Wings; Greektown Casino Hotel;}\\
\hline
\end{tabular}
\caption{Wikipedia Links Annotated in Tweets}
\label{table:wikilinks}
\end{table}




\section{Conclusion and Future Work}
\label{sec:conclusion}

As the role of social media continues to expand in emergency situations, the location of online users will play a crucial role in organizing relief efforts and disseminating information. In this paper, we have presented an approach that leverages Wikipedia to estimate the location of Twitter users. With a preliminary evaluation we have showed that the approach gives accuracy of 30\% for 935 users selected randomly from existing datasets. While the current approaches (network-based and content-based) require a significant amount of training data for predicting users' locations, we have introduced an alternative that can perform the same task by using crowd-sourced background knowledge. 

In this work, we used \textit{ICF} to identify the discriminating power of an internal link. In future, we would like to use other scoring techniques to filter out seemingly irrelevant internal links from our consideration. In particular we would like to use the graph structure of Wikipedia and semantic types of Wikipedia pages, to identify groups of internal links that display a stronger relationship to the city.

%We also acknowledge the limitation of this approach to be the coverage of Wikipedia., i.e. cities that are not present in Wikipedia will be ignored by our approach. We intend to explore other geo-datasets on LOD that can provide us with appropriate information to adapt to our approach.      
%In future we would like to explore other scoring techniques for entities for both \begin{inparaenum} \item creating background knowledge for cities and \item entities scoring from users' tweets \end{inparaenum}. Specifically, the creating of background knowledge presently uses \textit{ICF} which reflects the discriminative ability of the entity. However, we need to focus the usage of the entity for a particular city which is yet to be explored. We also acknowledge the limitation of this approach to be the coverage of Wikipedia., i.e. cities that are not present in Wikipedia will be ignored by our approach. We intend to explore other geo-datasets on LOD that can provide us with appropriate information to adapt to our approach.      
 



\bibliographystyle{plain}
\bibliography{smile_2014}
\end{document}