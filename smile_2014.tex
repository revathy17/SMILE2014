%%%%%%%%%%%%%%%%%%%%%%% file typeinst.tex %%%%%%%%%%%%%%%%%%%%%%%%%
%
% This is the LaTeX source for the instructions to authors using
% the LaTeX document class 'llncs.cls' for contributions to
% the Lecture Notes in Computer Sciences series.
% http://www.springer.com/lncs       Springer Heidelberg 2006/05/04
%
% It may be used as a template for your own input - copy it
% to a new file with a new name and use it as the basis
% for your article.
%
% NB: the document class 'llncs' has its own and detailed documentation, see
% ftp://ftp.springer.de/data/pubftp/pub/tex/latex/llncs/latex2e/llncsdoc.pdf
%
%%%%%%%%%%%%%%%%%%%%%%%%%%%%%%%%%%%%%%%%%%%%%%%%%%%%%%%%%%%%%%%%%%%


\documentclass[runningheads,a4paper]{llncs}

\usepackage{amssymb}
\setcounter{tocdepth}{3}
\usepackage{graphicx}
\usepackage{graphics}
\usepackage{multirow}
\usepackage{color}
\usepackage{paralist}
\usepackage{url}
\usepackage{subfigure}
\usepackage{listings} 
\usepackage{booktabs}

\lstloadlanguages{html,xml}
\definecolor{grey}{rgb}{0.9,0.9,0.9} 
\lstset{
        tabsize=2, 
        frame=single, 
        breaklines=true, 
        basicstyle=\footnotesize\ttfamily,
        backgroundcolor=\color{grey},
        xleftmargin=0mm,
        xrightmargin=0mm,
        captionpos=b
}

\usepackage{url}
\urldef{\mailsa}\path|{revathy,pavan,amit}@knoesis.org|    
\newcommand{\keywords}[1]{\par\addvspace\baselineskip
\noindent\keywordname\enspace\ignorespaces#1}

\graphicspath{./images/}
\DeclareGraphicsExtensions{.pdf,.jpeg,.png}	

\usepackage{color} 
\definecolor{gray}{rgb}{0.7,0.7,0.7}
\definecolor{blue}{rgb}{0.1,0.1,0.5}
\newcommand{\attention}[1]{{\color{red}\textbf{#1}}}
%\newcommand{\comment}[1]{\textit{\color{blue}#1}}
%\newcommand{\comment}[2]{{\color{blue}\textit{\textbf{#1}}(\textit{#2})}}
\newcommand{\comment}[2]{}

%\newcommand{\softdelete}[1]{{\color{gray}\textit{#1}}}
\newcommand{\softdelete}[1]{}

\newcommand{\uri}[1]{\texttt{#1}}
\newcommand{\literal}[1]{`\textit{#1}'}
\newcommand{\BibTeX}{{\sc Bib}\TeX}
\lstset{basicstyle=\small}
\begin{document}

\mainmatter  % start of an individual contribution

% first the title is needed
\title{Geo Tagging Twitter Users using Wikipedia}

% a short form should be given in case it is too long for the running head
\titlerunning{Geo Tagging Twitter Users using Wikipedia}

\author{Revathy Krishnamurthy\and Pavan Kapanipathi\and Amit Sheth}

\authorrunning{R.Krishnamurthy\and P.Kapanipathi\and A.Sheth}

\institute{
Kno.e.sis Center, CSE Department\\
Wright State University, Dayton, OH - USA \\
}

%\toctitle{Semantic Multicasting}
\tocauthor{R.Krishnamurthy, P.Kapanipathi, A.Sheth}
\maketitle

\begin{abstract}
As more and more people are taking to microblogging networks, like Twitter, as a primary source of their communication with the rest of the world, the analysis of user generated content has become increasingly significant for crisis management. To make the contents of a tweet actionable, we need to be able to determine the location of the user. A recent study has shown that approximately only 3.17\% of tweets are tagged with the user location. As tweets are generally informal in nature and contain many acronyms and slang words, researchers have focussed on using statistical methods for identification of words with a strong geographic scope and then use these words to identify the location of a user. But the biggest challenge in this approach is the requirement of a training dataset and creation of the statistical model, which can be a time consuming process. To this end, we propose an approach that uses Wikipedia as a background knowledge to analyse tweets in order to predict the location of the users. The main advantage of the proposed approach is that the use of Wikipedia eliminates the need for a training dataset. We show that initial tests with this approach allows us to locate 30\% of users within 100 miles of their actual location.

%deleted:https://pressroom.usc.edu/twitter-and-privacy-nearly-one-in-five-tweets-divulge-user-location-through-geotagging-or-metadata
\end{abstract}
\section{Introduction}
\label{sec:Introduction}

%use of twitter in crisis management
The power of social media was demonstrated during Hurricane Sandy when more than 20 million tweets related to the hurricane were posted in a span of three days. It was reported that the volume of tweets doubled in these days as compared to the previous two days. 35\% of these tweets contained news from media channels, information from government sources and eyewitness accounts\footnote{\url{http://www.journalism.org/2012/11/06/hurricane-sandy-and-twitter/}}. This kind of extensive use of social networking platforms during emergency situations has paved the way for new areas of research which focus on leveraging these platforms for disaster management~\cite{purohit2013emergency}. 

In improving emergency response using information from Twitter, the location of an online user plays an important role. However, only  3.17\% of tweets are tagged with geographic coordinates~\cite{morstatter2013sample}. As tweets are generally informal in nature and contain many acronyms and slang words, researchers have focussed on using statistical methods for identification of words with a strong geographic scope and then use these words to identify the location of a user.
%Current approaches to predict location of Twitter users based on their tweets, focus on building statistical models.
In the event of a disaster, to identify user location in real time, we need an approach that can be easily adapted to any geographic location. In order to overcome this challenge, we present an approach that utilizes Wikipedia as a source of background knowledge to predict users' location based on their online content. Briefly, our approach uses the graph structure of Wikipedia to find entities with a local geographic scope. The presence of these entities in users' tweets help estimate their location. Our intuition is that, more the users talk about entities with a local geographic scope, more are their chances of belonging to that location. Preliminary evaluation of our approach with a random sample from the dataset shared by Cheng et al\cite{cheng2010you} has shown promise and performs better than their baseline.   

In the rest of this paper, we will first discuss the related work on location prediction of Twitter users in Section~\ref{sec:RelatedWork}, followed by a detailed explanation of our approach in Section~\ref{sec:approach}. Section~\ref{sec:evaluation} discusses a preliminary evaluation of our approach and the paper concludes with a discussion on future work in Section~\ref{sec:conclusion}. 
\section{Related Work}
\label{sec:RelatedWork}
There have been two main approaches in addressing the problem of location identification of a twitter user: \begin{inparaenum}[(1)] 
\item Using the content of the tweets: based on the premise that the online content of a user is influenced by the geographical location of the user
\item Using the network information of the user: based on the assumption that the location of the people in a user's network and their online interaction can be used to determine the user's location.
\end{inparaenum}

Content-based location detection relies on a significantly large training dataset to build a statistical model that identifies words with a local scope. Use of these words in tweets are then used to narrow down the location of any user. Cheng et al. \cite{cheng2010you} proposed a probabilistic framework for estimating a Twitter user's city-level location based on the content of approximately 1000+ tweets of each user. The locality of terms was determined by its spatial variation across united states. Their approach on a test dataset of 130689 users with 1000+ tweets each, could locate 51\% of the users within 100 miles. The disadvantage of this approach was the assumption that a "term" is spatially significant to only one location/city. This challenge was addressed by Chang et al.~\cite{chang2012phillies} by modeling the variations as a Gausian mixture model. Their tests on the same test dataset showed an accuracy (within 100 miles) of 0.499 with 509.3 miles of average error distance.
%\cite{ferrara2012web} created language models at different granularity levels from zip code to country level using a training dataset of 5.8 million geotagged tweets.
 %They reported their results on two datasets - SPRITZER containing 5\% of the public twitter stream of 4 weeks and FIREHOSE containing 700,000 tweets from the Twitter Firehose. At the city-level, they reported an accuracy of 65.7\% and 29.8\% on the SPRITZER and the FIREHOSE dataset respectively.

Network based solutions requires the network information of a given user. McGee et al. \cite{mcgee2013location} train an SVM classifier with features based on the information of users' followers-followees who have their location information available. They tested their approach on a random sample of 1000 users and reported 50.08\%accuracy at the city level. However, the limitation of a network-based approaches is the availability of location information of people in the appropriate user's network.   
%Their training dataset consisted of 1.6 million twitter users and their network information. On a test dataset of 249,584 users, they reported 63.9\% accuracy in determining the location within 25 miles. Rout et al. \cite{rout2013s} formulated this task as a classification task and trained an SVM classifier on twitter users with known location, to use a person's social network to locate them. Their training dataset contained 200,000 twitter users. They tested their approach on a random sample of 1000 users and reported 50.08\%accuracy at the city level.

The above mentioned approaches require prior training dataset (of either the content or network), which can be a bottleneck during disaster management. Our goal is to overcome this requirement of training data for each city by leveraging Wikipedia as the knowledge source.
%citeInferring the location of twitter messages based on user relationships.
%\input{background.tex}
%\input{architecture.tex}
%\section{Approach}
\label{sec:approach}
\begin{wrapfigure}[13]{l}{45mm}
\centering
\includegraphics[width = 5cm]{images/architecture-2.jpg}
%\vspace*{-0.75em}
\caption{Architecture}
\label{fig:architecture}
\end{wrapfigure}
\vspace*{-1.00em}
Previous research~\cite{bo2012geolocation,cheng2010you} have established that the content of a user's posts reflects his/her geographical location. Using the same intuition we propose an approach that uses Wikipedia as the background knowledge. An overview of the approach is shown in Figure~\ref{fig:architecture}. It comprises of three components \begin{inparaenum}[(1)]\item Tweets Annotator: Extracts Wikipedia entities from a user's tweets, \item Background Knowledge Generator: Generates background knowledge for each city using Wikipedia \item Location Predictor: Utilizes the output of \textit{Tweets Annotator} with \textit{Background Knowledge} to predict the location of the user. \end{inparaenum} 
%We propose to use the information available in Wikipedia to establish entities that are most representative of a given location. Wikipedia is a large encyclopedia containing dedicated pages for cities. Proportional to the size of the city, it Wikipedia page generally contains a variety of information about the city like it geography, culture, sports team, cityscape etc. Our hypothesis is that by correlating the occurrence of city specific entities from Wikipedia in a user's tweets, we can estimate the location of the user

%\subsection{Dataset} (This should go to the Evaluation section)
%We randomly selected 600 users containing 1000+ tweets each, from the dataset made publicly available by Cheng et al\cite{cheng2010you}. Our dataset has users from 48 states across US. 
\subsection{Annotation of Users' Tweets}
Derczynski et.al, in their latest work~\cite{derczynski2013} have compared three state of art \textit{entity recognition and linking} systems for tweets. The systems compared with corresponding \textit{Precision, Recall and F-Measures} are \begin{inparaenum} \item Dbpedia Spotlight~\cite{mendes2011dbpedia} (P=20.1, R=47.4, F=28.3), \item Zemanta\footnote{\url{http://developer.zemanta.com/}} (P=57.5, R=31.8, F=\textbf{41.0}) and \item TextRazor\footnote{\url{http://www.textrazor.com/technology}} (P=\textbf{64.6}, R=26.9, F=38.0) \end{inparaenum}. We used Zemanta\footnote{\url{http://developer.zemanta.com/docs/suggest/}} because of its relatively superior performance and also because of their rate limit extension (10,000 per day) provided for research purposes, on request\footnote{We thanks Zemanta for their support.}. 

\subsection{Creation of Background Knowledge}
Wikipedia is a large encyclopedia containing dedicated pages for geographical locations. Proportional to the size of the location, its Wikipedia page generally contains a variety of information about the place like its geography, culture, sports team, cityscape etc.

Links to internal Wikipedia pages from a given page are an important feature of all Wikipedia pages. The aim of these links is to increase the understanding of a user about the given page. For instance, the Wikipedia page of \textit{Boston, Massachusetts} \footnote{http://en.wikipedia.org/wiki/Boston} mentions the \textit{Boston Red Sox}, in the Sports section. It also provides a hyperlink to Boston Red Sox, that allows the user to navigate to the Wikipedia page of \textit{Boston Red Sox}. We base our approach on the assumption that these internal links share varying degrees of relevance to the Wikipedia page of the city. As in the previous example, the Wikipedia page of Boston also contains an internal link to \textit{Major League Baseball} which would be less representative of Boston than the \textit{Boston Red Sox}. 

To create our knowledgebase, we selected 1670 cities in the United States of America having population greater than 20000. The entire collection of Wikipedia is available for download\footnote{http://en.wikipedia.org/wiki/Wikipedia:Database\_download}. We use the dump dated 14-Feb-2014 to extract the internal links from the Wikipedia pages of all the cities in our dataset. Figure 1 shows the distribution of the count of internal links among all the city pages. From our dataset, \textit{Pittsburgh} had 2684 as the largest count of internal links and \textit{Round Lake Beach, Illinois} had 33 as the smallest count of internal links.

\subsubsection{Scoring city-specific Entities}
Given a set of internal links for a city, we score each link to determine the degree of its relevance to the city. The more a given internal link is common to the cities in our dataset, the less it maybe relevant to one particular city. For example, in our dataset of 1650 cities, an internal link to the Wikipedia page of \textit{Barack Obama} appears 105 times as opposed to \textit{Southern California} and \textit{Golden Gate Bridge} which appear 50 and 6 times respectively. 

Mendes et al. \cite{mendes2011dbpedia} proposed \textit{Inverse Candidate Frequency} for the task of entity disambiguation in DBPedia Spotlight. The idea behind ICF is that "a word commonly co-occuring with many resources is less discriminative overall". We use this intuition to identify the discriminative ability of an internal link with respect to a city. Let C be the set of cities in our dataset. Let I be the set of internal links for a city c $\in$ C. The ICF of an internal link i $\in$ I, that appears in \textit{n} cities, is defined as:
\begin{equation}
	ICF(i) = \log |C|- \log n
\end{equation}

\subsection{Location Estimation}
We used Zemanta\footnote{http://www.zemanta.com} to annotate tweets. It maps entities in the input text to Wikipedia pages.

For a user U, let $T_{u}$ be the set of their tweets, $Z_{u}$ = \{$z_{1}$,$z_{2}$,...,$z_{k}$\} be the set of entities annotated by Zemanta that map to a Wikipedia url. Let |$z_{k}$| represent the cardinality $z_{k}$ in $T_{u}$. Let C be the set of cities in out dataset and $\forall$ $c_{j}$ $\in$ C, let $L$ be the set of its internal wiki links where \textit{ICF}($l_{i}$) is the score $\forall$ $l_{i}$ $\in$ L.

For the user U we compute the score of each city in our set as:
\begin{equation}
	Score(c_{j}) = \sum_{i=1}^I |l_{i}| \times ICF(l_{i})\qquad  \forall l_{i} \in Z_{u}
\end{equation}

We tag the city with the maximum score as the location of the user. (PAVAN: User argmax/argmin here -- make it more a mathematical notation).




 
%\section{Conclusion and Future Work}
\label{sec:conclusion}

As the role of social media continues to expand in emergency situations, the location of online users will play a crucial role in organizing relief efforts and disseminating information. In this paper, we have presented an approach that leverages Wikipedia to estimate the location of Twitter users. With a preliminary evaluation we have showed that the approach gives accuracy of 30\% for 935 users selected randomly from existing datasets. While the current approaches (network-based and content-based) require a significant amount of training data for predicting users' locations, we have introduced an alternative that can perform the same task by using crowd-sourced background knowledge. 

In this work, we used \textit{ICF} to identify the discriminating power of an internal link. In future, we would like to use other scoring techniques to filter out seemingly irrelevant internal links from our consideration. In particular we would like to use the graph structure of Wikipedia and semantic types of Wikipedia pages, to identify groups of internal links that display a stronger relationship to the city.

%We also acknowledge the limitation of this approach to be the coverage of Wikipedia., i.e. cities that are not present in Wikipedia will be ignored by our approach. We intend to explore other geo-datasets on LOD that can provide us with appropriate information to adapt to our approach.      
%In future we would like to explore other scoring techniques for entities for both \begin{inparaenum} \item creating background knowledge for cities and \item entities scoring from users' tweets \end{inparaenum}. Specifically, the creating of background knowledge presently uses \textit{ICF} which reflects the discriminative ability of the entity. However, we need to focus the usage of the entity for a particular city which is yet to be explored. We also acknowledge the limitation of this approach to be the coverage of Wikipedia., i.e. cities that are not present in Wikipedia will be ignored by our approach. We intend to explore other geo-datasets on LOD that can provide us with appropriate information to adapt to our approach.      
 




\bibliographystyle{plain}
\bibliography{smile_2014}
\end{document}