\section{Conclusion and Future Work}
\label{sec:conclusion}

As the role of social media continues to expand in emergency situations, the location of online users will play a crucial role in organizing relief efforts and disseminating information. In this paper, we have presented an approach that leverages Wikipedia to estimate the location of Twitter users. With a preliminary evaluation we have showed that the approach gives accuracy of 30\% for 935 users selected randomly from existing datasets. While the current approaches (network-based and content-based) require a significant amount of training data for predicting users' locations, we have introduced an alternative that can perform the same task by using crowd-sourced background knowledge. 

In this work, we used \textit{ICF} to identify the discriminating power of an internal link. In future, we would like to use other scoring techniques to filter out seemingly irrelevant internal links from our consideration. In particular we would like to use the graph structure of Wikipedia and semantic types of Wikipedia pages, to identify groups of internal links that display a stronger relationship to the city.

%We also acknowledge the limitation of this approach to be the coverage of Wikipedia., i.e. cities that are not present in Wikipedia will be ignored by our approach. We intend to explore other geo-datasets on LOD that can provide us with appropriate information to adapt to our approach.      
%In future we would like to explore other scoring techniques for entities for both \begin{inparaenum} \item creating background knowledge for cities and \item entities scoring from users' tweets \end{inparaenum}. Specifically, the creating of background knowledge presently uses \textit{ICF} which reflects the discriminative ability of the entity. However, we need to focus the usage of the entity for a particular city which is yet to be explored. We also acknowledge the limitation of this approach to be the coverage of Wikipedia., i.e. cities that are not present in Wikipedia will be ignored by our approach. We intend to explore other geo-datasets on LOD that can provide us with appropriate information to adapt to our approach.      
 

